\documentclass[10pt, a4paper,spanish]{article}
\usepackage[utf8]{inputenc}

\usepackage{lipsum} % Package to generate dummy text throughout this template
\usepackage{varwidth}

\usepackage[T1]{fontenc} % Use 8-bit encoding that has 256 glyphs
\usepackage{microtype} % Slightly tweak font spacing for aesthetics

\usepackage[hmarginratio=1:1,top=32mm,columnsep=20pt]{geometry} % Document margins
\usepackage{multicol} % Used for the two-column layout of the document
\usepackage[hang, small,labelfont=bf,up,textfont=it,up]{caption} % Custom captions under/above floats in tables or figures
\usepackage{booktabs} % Horizontal rules in tables
\usepackage{float} % Required for tables and figures in the multi-column environment - they need to be placed in specific locations with the [H] (e.g. \begin{table}[H])
\usepackage{hyperref} % For hyperlinks in the PDF

\usepackage{lettrine} % The lettrine is the first enlarged letter at the beginning of the text
\usepackage{paralist} % Used for the compactitem environment which makes bullet points with less space between them

\usepackage{abstract} % Allows abstract customization
\renewcommand{\abstractnamefont}{\normalfont\bfseries} % Set the "Abstract" text to bold
\renewcommand{\abstracttextfont}{\normalfont\small\itshape} % Set the abstract itself to small italic text

\usepackage{titlesec} % Allows customization of titles
\renewcommand\thesection{\Roman{section}} % Roman numerals for the sections
\renewcommand\thesubsection{\Roman{subsection}} % Roman numerals for subsections
\titleformat{\section}[block]{\large\scshape\centering}{\thesection.}{1em}{} % Change the look of the section titles
\titleformat{\subsection}[block]{\large}{\thesubsection.}{1em}{} % Change the look of the section titles

\usepackage{fancyhdr} % Headers and footers
\pagestyle{fancy} % All pages have headers and footers
\fancyhead{} % Blank out the default header
\fancyfoot{} % Blank out the default footer
\fancyhead[C]{Sergio García Prado $\bullet$ Febrero 2016 $\bullet$ Propuesta Individual} % Custom header text
\fancyfoot[RO,LE]{\thepage} % Custom footer text

%----------------------------------------------------------------------------------------
%	TITLE SECTION
%----------------------------------------------------------------------------------------

\title{\vspace{-15mm}\fontsize{24pt}{10pt}\selectfont\textbf{Propuesta Individual}} % Article title

\author{
\large
\textsc{Sergio García Prado}\\[2mm] % Your name
\normalsize Universidad de Valladolid \\ % Your institution
\vspace{-5mm}
}
\date{}

%----------------------------------------------------------------------------------------

\begin{document}

	\maketitle % Insert title

	\thispagestyle{fancy} % All pages have headers and footers

%----------------------------------------------------------------------------------------
%	ABSTRACT
%----------------------------------------------------------------------------------------

	\begin{abstract}
		\noindent Página Web 
	\end{abstract}

		\section{Proposición de Compañeros del grupo de trabajo}
			
			\begin{center}
				\begin{varwidth}{\textwidth}
					\begin{itemize}
						\item Amigo Alonso, Alberto
						\item Delgado Álvarez, Sergio
						\item Fernández Angulo, Oscar
					\end{itemize}
				\end{varwidth}
			\end{center}
			
		\section{Descripción del Problema}
		
			\paragraph{}
			Cada vez más existen voces críticas respecto a la globalización, que busca la reducción de costes sin tener en cuenta las consecuencias sociales de dichas acciones. Hoy en día los productores venden sus productos a intermediarios que después venden estos a miles de kilómetros de distancia. Este modelo de negocio premia a los intermediarios, que son quienes reciben la mayor parte del beneficio, provocando que los productores tengan que vender sus productos a precios que rozan los costes de producción, siendo algunas veces inferiores a estos. Muchas veces esto conlleva problemas sociales en la zona ya que muchos productores se ven obligados a despedir a sus trabajadores para poder obtener un mayor beneficio que siga haciendo viable el negocio. Esto provoca que el consumo en la zona también disminuya por culpa del desempleo.
			
			\paragraph{}
			Una alternativa a este modelo es intentar comprar productos de la zona, lo cual reduce costes de intermediarios por lo cual los precios de estos tienden a ser menores para el cliente y los beneficios se aumentan para los productores. Existen varios problemas respecto a este enfoque como que la región no puede producir todos los productos demandados por el cliente o que muchas veces los clientes tienen dificultades para contactar con los productores.
			
			\paragraph{}
			Para exponer esta idea nos centraremos en los productos alimenticios, pero esto es extrapolable a cualquier tipo de producto. La solución que propondremos deberá ser capaz de cumplir un conjunto de funcionalidades o tareas:
			\begin{itemize}
				\item Los productores podrán ofertar sus productos para que los clientes los puedan encontrar fácilmente.
				\item Los productos tendrán un sistema de valoración de calidad basado en las experiencias de los clientes.
				\item Los clientes tendrán un catálogo de productos que poder comprar.
				\item Los clientes tendrán un mecanismo para encontrar fácilmente los productos que deseen.
				\item El sistema premiará la venta de productos locales.
			\end{itemize}
			
			\paragraph{}
			La idea de comprar productos alimenticios por internet no es algo nuevo por lo que ya existen diferentes alternativas que se dedican a ello pero todos ellos se centran en ser también quienes lleven el producto a casa.
			
		\section{Usuarios Objetivo}
		
			\paragraph{}
			Página Web

		\section{Borrador de la Solución}
		
			\paragraph{}
			Página Web




\end{document}
