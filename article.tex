\documentclass[10pt, a4paper,spanish]{article}
\usepackage[utf8]{inputenc}

\usepackage{lipsum} % Package to generate dummy text throughout this template
\usepackage{varwidth}
\usepackage{hyperref}
\usepackage{graphicx}

\usepackage[T1]{fontenc} % Use 8-bit encoding that has 256 glyphs
\usepackage{microtype} % Slightly tweak font spacing for aesthetics

\usepackage[hmarginratio=1:1,top=32mm,columnsep=20pt]{geometry} % Document margins
\usepackage[hang, small,labelfont=bf,up,textfont=it,up]{caption} % Custom captions under/above floats in tables or figures
\usepackage{booktabs} % Horizontal rules in tables
\usepackage{float} % Required for tables and figures in the multi-column environment - they need to be placed in specific locations with the [H] (e.g. \begin{table}[H])
\usepackage{hyperref} % For hyperlinks in the PDF

\usepackage{lettrine} % The lettrine is the first enlarged letter at the beginning of the text
\usepackage{paralist} % Used for the compactitem environment which makes bullet points with less space between them

\usepackage{abstract} % Allows abstract customization
\renewcommand{\abstractnamefont}{\normalfont\bfseries} % Set the "Abstract" text to bold
\renewcommand{\abstracttextfont}{\normalfont\small\itshape} % Set the abstract itself to small italic text

\usepackage{titlesec} % Allows customization of titles
\renewcommand\thesection{\Roman{section}} % Roman numerals for the sections
\renewcommand\thesubsection{\Roman{subsection}} % Roman numerals for subsections
\titleformat{\section}[block]{\large\scshape\centering}{\thesection.}{1em}{} % Change the look of the section titles
\titleformat{\subsection}[block]{\large}{\thesubsection.}{1em}{} % Change the look of the section titles

\usepackage{fancyhdr} % Headers and footers
\pagestyle{fancy} % All pages have headers and footers
\fancyhead{} % Blank out the default header
\fancyfoot{} % Blank out the default footer
\fancyhead[C]{ Marzo 2016 $\bullet$ Propuesta de Grupo} % Custom header text
\fancyfoot[RO,LE]{\thepage} % Custom footer text

%----------------------------------------------------------------------------------------
%	TITLE SECTION
%----------------------------------------------------------------------------------------

\title{\vspace{-15mm}\fontsize{24pt}{10pt}\selectfont\textbf{Propuesta de Grupo}} % Article title

\author{
\large
\textsc{Alberto Amigo Alonso\textsubscript{25\%}}\\[2mm] % Your name
\textsc{Sergio Delgado Álvarez\textsubscript{25\%}}\\[2mm] % Your name
\textsc{Sergio García Prado\textsubscript{25\%}}\\[2mm] % Your name
\textsc{Oscar Fernández Angulo\textsubscript{25\%}}\\[2mm] % Your name
\normalsize Universidad de Valladolid \\ % Your institution
\vspace{-5mm}
}
\date{}

%----------------------------------------------------------------------------------------

\begin{document}

	\maketitle % Insert title

	\thispagestyle{fancy} % All pages have headers and footers

%----------------------------------------------------------------------------------------
%	ABSTRACT
%----------------------------------------------------------------------------------------

	\begin{abstract}
		\noindent Servicio web destinado a permitir a remitentes y destinatarios ofertar envíos para que los transportistas sean capaces de encontrarlos permitiendo a todos los usuarios monitorizarlos.
	\end{abstract}

%----------------------------------------------------------------------------------------
%	TEXT
%----------------------------------------------------------------------------------------
		\section{Justificación de la elección}

			\paragraph{}
			Como propuesta de proyecto de grupo se ha decidido continuar con la idea del servicio de transporte de paquetes por el reto que supone la implementación de la elevada cantidad de contenido dinámico. Este se encuentra en todo el proceso de creación de un nuevo envío por parte de un usuario, así como de la oferta de los transportistas y la elección del usuario de entre todas las ofertas.

			\paragraph{}
			Los cambios que se han realizado desde la primera propuesta han sido tanto del modelo de negocio de la empresa como de la implementación del diseño del sitio web. En cuanto al modelo de negocio, se ha añadido el rol de cliente-receptor (no realiza un envio, sino que requiere de alguien le recoja un paquete y se lo envie). Los cambios sobre el diseño se pueden ver en el documento desarrollado, aunque se enumeran a continuación:

				\begin{enumerate}
					\item Se facilita a los transportistas las rutas de los envios por los que pueden pujar mediante mapas, para que puedan diseñar sus propias rutas de transporte y aprovechar el camino para realizar varios pedidos.

					\item Se diferencian más rotundamente los roles de transportista y remitente, especializando los comportamientos que pueden realizar cada uno.

					\item La página de inicio que se muestra a los usuarios es directamente el historial de envios que tienen pendientes/finalizados. Esto proporciona agilidad a los usuarios en la experiencia de navegación por la web.

					\item En el historial de envios se usa un código de colores para indicar de una forma visual el estado de los envios. Por ejemplo Verde - Envio completado con exito, Amarillo - Pendiente de asignación.

					\item Se pasa del modelo de tablas que representan cada envio por un modelo de tarjetas desplegables. Cada tarjeta desplegable puede albergar mucha mas infomación que la tabla, por lo que se comprime mas la información sin llegar a resultar un problema de navegación.

					\item Se incorpora la plataforma de marcado de hitos. Cuando un remitente realiza un pago, marca la acción para que el transportista sepa que debe verificarlo.
				\end{enumerate}



		\section{Descripción del Problema}

			\paragraph{}
			El transporte es una actividad del sector terciario, entendida como el desplazamiento de objetos o personas de un lugar (punto de origen) a otro (punto de destino) en un vehículo (medio o sistema de transporte) que utiliza una determinada infraestructura (red de transporte). Esta ha sido una de las actividades terciarias que mayor expansión ha experimentado a lo largo de los últimos dos siglos, debido a la industrialización; al aumento del comercio y de los desplazamientos humanos tanto a escala nacional como internacional; y los avances técnicos que se han producido y que han repercutido en una mayor rapidez, capacidad, seguridad y menor coste de los transportes. \cite{wikipedia_transporte}

			\paragraph{}
			Nosotros nos centraremos en el transporte de mercancías para la realización de esta propuesta. En este sector existen tres roles bien diferenciados: el \textbf{remitente}, el \textbf{transportista} y el \textbf{destinatario}. El modelo de negocio actual se basa en empresas intermediarias denominadas \textit{agencias de transporte} cuya labor es poner en contacto remitentes con transportistas que lleven la mercancía hasta su destino.

			\paragraph{}
			Dependiendo de la manera en que se contrate el transporte existen dos formas de pago: por parte del remitente, del destinatario o respecto de una determinada proporción. Actualmente el método más utilizado es el de pago por parte del destinatario pero aún así existen casos especiales en los que no. También existen dos métodos de calcular el precio del transporte: aplicando una tarifa fija (lo cual muchas veces infla los costes de manera innecesaria) o teniendo en cuenta la distancia recorrida desde el origen hasta el destino. Uno de los problemas actuales es el de calcular penalizaciones y recompensas por el tiempo de transporte así como el periodo de tiempo que transcurre durante la carga y descarga de la mercancía.

			\paragraph{}
			La solución que se propone es crear un servicio encargado de conectar a transportistas con remitentes y destinatarios de manera eficiente y automática tratando de minimizar los costes entre intermediarios, dando lugar a un mayor beneficio para el transportista y un menor coste para el remitente/destinatario. En la siguiente sección se describirá más detalladamente tanto las acciones como la descripción de cada uno de los usuarios objetivo.

			\paragraph{}
			El servicio seguirá una idea similar a la de Uber \copyright\  con el transporte de pasajeros pero aplicado al campo de las mercancías (siempre siguiendo la vigente normativa). Ya existen algunas start-ups que pretenden hacerse un hueco en este nicho de mercado entre las que destacan Convoy \copyright\ y Trucker Path \copyright\ las cuales se basan en una aplicación para sistemas móviles. \cite{expansion_uber_transporte}

			\paragraph{}
			Los usuarios del sistema podran realizar las siguientes acciones:

			\begin{itemize}

				\item Los usuarios se registran en el sistema diferenciando su rol (remitente/destinatario o transportista)

				\item Los remitentes pueden crear envíos en el sistema para que los transportistas les oferten un precio por llevarlo a cabo.

				\item Los remitentes/destinatarios seleccionarán al transportita que deseen de entre los que han aceptado la oferta.

				\item Una vez confirmado el envío los remitentes/destinatarios pueden monitorizar el envío

				\item Los transportistas pueden buscar envíos que realizar introducciendo su posición geográfica y el sistema devolverá un listado con los más cercanos.

				\item Todos los usuarios del sistema pueden comentar en la página del envío para comunicarse entre sí.
			\end{itemize}



		\section{Usuarios Objetivo}

			\paragraph{}
			Los usuarios objetivo del servicio serán los siguientes:

			\begin{itemize}

				\item{\textbf{Remitente}}
				\newline
				El remitente será la persona encargada de crear el envío. Debido a ello es quien debe introducir todos los datos de dirección, tanto los suyos para que el transportista pueda recoger el envío, como los del destinatario para que este pueda llegar hasta él. Además será el que seleccione el método de pago para luego proceder al pago según la opción seleccionada, bien por parte del propio remitente o por parte del destinatario. El precio no es algo que se fijará "a priori", sino que será después cuando los transportistas encuentren el envío cuando pujarán por un precio de transporte. Tendrá la capacidad de monitorizar el estado del envío.

				\item{\textbf{Transportistas}}
				\newline
				El transportista es la persona encargada de llevar el envío desde el lugar de origen hasta el de destino. El sistema se encargará de permitir al transportista encontrar envíos disponibles en la base de datos y "pujar" un precio por el servicio para que después el encargado de pagar por ello seleccione la opción que más le convenga.

				\item{\textbf{Destinatario}}
				\newline
				El destinatario es quién recibirá el envío. En el caso de ser el encargado del pago tendrá la responsabilidad también de elegir el transportista. En caso contrario su única labor será la de fijar el horario preferido de entrega a la vez que monitorizar el estado del envío.

			\end{itemize}
		\section{Borrador de la Solución}

			\paragraph{}
			El servicio web estara estructurado para navegación por pestañas y vistas emergentes que se superpongan encima del resto del contenido de la página preferiblemente sin crear una nueva ventana emergente del navegador.

			\paragraph{}
			A continuación procederemos a describir el conjunto de vistas que formarán el servicio web. Dado que existen dos roles bien diferenciados (remitente/destinatario y transportista) las vistas que sean comunes a los dos pero tengan pequeñas diferencias se desribirán juntas explicando dichas variaciones.

			\begin{figure}[H]
				\centering
				\begin{minipage}[b]{0.7\textwidth}
					\includegraphics[width=\textwidth]{res/sketch_login.png}
				\end{minipage}
			\end{figure}

			\paragraph{}
			La vista de \textbf{login} será lo primero que el usuario vea al acceder al sitio web, en esta parte se identificará o creará una cuenta en el caso de que no la tuviera todavía.



			\begin{figure}[H]
				\centering
				\begin{minipage}[b]{0.49\textwidth}
					\includegraphics[width=\textwidth]{res/sketch_create_account.png}
				\end{minipage}
				\begin{minipage}[b]{0.49\textwidth}
					\includegraphics[width=\textwidth]{res/sketch_create_account_carrier.png}
				\end{minipage}
			\end{figure}

			\paragraph{}
			En la vista de \textbf{crear cuenta} es donde el usuario introducirá sus datos personales. Esta parte será común para los dos roles. Además habrá un checkbutton que el usuario deberá marcar en el caso de que sea transportista. Entonces en la vista aparecerán nuevos campos donde este deberá introducir las caracteristicas de su método de transporte (tamaño, peso máximo, tipo de envíos que puede transportar, etc). En este punto el sistema ya tendrá diferenciados los dos roles.

			\begin{figure}[H]
				\centering
				\begin{minipage}[b]{0.49\textwidth}
					\includegraphics[width=\textwidth]{res/sketch_home.png}

				\end{minipage}
				\begin{minipage}[b]{0.49\textwidth}
					\includegraphics[width=\textwidth]{res/sketch_home_carrier.png}

				\end{minipage}
			\end{figure}

			\paragraph{}
			La vista que de \textbf{página principal} del sitio web. En ella se le mostrará información acerca de los envíos en curso, los que están pendientes de confirmación (para que los confirme en el caso del cliente y para saber si han sido confirmados y debe dirigirse al lugar de recogida en el caso del transportista). Además también habrá una sección destinada a dar la opinión acerca de anteriores envíos en el caso de los clientes. En el de los transportistas esta sección estará compuesta por las valoraciones que los clientes le han dado.


			\begin{figure}[H]
				\centering
				\begin{minipage}[b]{0.49\textwidth}
					\includegraphics[width=\textwidth]{res/sketch_deliveries.png}

				\end{minipage}
				\begin{minipage}[b]{0.49\textwidth}
					\includegraphics[width=\textwidth]{res/sketch_deliveries_carrier.png}

				\end{minipage}
			\end{figure}

			\paragraph{}
			La vista de \textbf{envíos} será la que permitirá al usuario obtener el historial de envíos posibilitandose un buscador para filtrar la información además de unas pestañas para filtrar entre paquetes enviados y recibidos. Esta página también será común para clientes y transportistas con la diferencia de que los transportistas no tendrán la posibilidad de seleccionar entre paquetes enviados y recibidos.


			\begin{figure}[H]
				\centering
				\begin{minipage}[b]{0.49\textwidth}
					\includegraphics[width=\textwidth]{res/sketch_stats.png}

				\end{minipage}
				\begin{minipage}[b]{0.49\textwidth}
					\includegraphics[width=\textwidth]{res/sketch_stats_carrier.png}

				\end{minipage}
			\end{figure}

			\paragraph{}
			La vista de \textbf{estadisticas} servirá para que los usuarios puedan obtener información gráficamente de manera rápida sobre envíos pasados. Tendrán un gráfico de líneas en el cual se mostrará la evolución del número de envíos respecto del tiempo. Además también tendrán otras métricas como el número de veces que han usado el mismo transportista en el caso del cliente y a la inversa en el caso del transportista. También podrán obtener métricas acerca de los orígenes y destinos a los que se han enviado pedidos, etc.

			\paragraph{}
			Las siguientes vistas serán de tipo emergente, es decir, se superpondrán al resto del contenido de la página. Si es posible se intentará hacer sin utilizar ventanas.


			\begin{figure}[H]
				\centering
				\begin{minipage}[b]{0.7\textwidth}
					\includegraphics[width=\textwidth]{res/sketch_create_delivery.png}

				\end{minipage}
			\end{figure}

			\paragraph{}
			La vista de \textbf{crear envío} será única para los clientes. En esta introducirán información acerca de qué quieren enviar, desde dónde hasta dónde, en qué fecha, etc. El cliente además tendrá la posibilidad de introducir imágenes del paquete que después el transportista podrá visualizar antes de solicitar realizar el envío.


			\begin{figure}[H]
				\centering
				\begin{minipage}[b]{0.7\textwidth}
					\includegraphics[width=\textwidth]{res/sketch_search_deliveries.png}

				\end{minipage}
			\end{figure}

			\paragraph{}
			La vista de \textbf{buscar envíos} será única para los transportistas. En esta vista los transportistas introducirán su ubicación actual o pulsarán en un botón que se la reconozca automáticamente y seguidamente el sistema les devolverá un listado los envíos disponibles más cercanos. Seguidamente el transportista podrá introducir un precio por el que está dispuesto a realizar el envío.


			\begin{figure}[H]
				\centering
				\begin{minipage}[b]{0.7\textwidth}
					\includegraphics[width=\textwidth]{res/sketch_delivery.png}

				\end{minipage}
			\end{figure}

			\paragraph{}
			La vista de \textbf{envío} será similar para los clientes y transportistas. En ella se mostrará información acerca del envío, así como el transcurso del mismo. También habrá posibilidad de escribir comentarios en ella para que el remitente, el destinatario y el transportista se puedan comunicar, lo que permitirá una  mayor satisfacción del servicio de transporte tanto por los clientes como por los transportistas.
%----------------------------------------------------------------------------------------
%	Bibliographic references
%----------------------------------------------------------------------------------------
	\begin{thebibliography}{9}

		\bibitem{wikipedia_transporte}
		Wikipedia. Transporte. \url{https://es.wikipedia.org/wiki/Transporte}

		\bibitem{expansion_uber_transporte}
		Expansión. Se busca el Uber del transporte de mercancías. \url{http://www.expansion.com/economia-digital/2015/10/29/5630ef8e46163f932a8b45d9.html}

	\end{thebibliography}


\end{document}
