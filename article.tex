\documentclass[10pt, a4paper,spanish]{article}
\usepackage[utf8]{inputenc}

\usepackage{varwidth}
\usepackage{graphicx}

\usepackage[T1]{fontenc} % Use 8-bit encoding that has 256 glyphs
\usepackage{microtype} % Slightly tweak font spacing for aesthetics

\usepackage[hmarginratio=1:1,top=32mm,columnsep=20pt]{geometry} % Document margins
\usepackage[hang, small,labelfont=bf,up,textfont=it,up]{caption} % Custom captions under/above floats in tables or figures
\usepackage{float} % Required for tables and figures in the multi-column environment - they need to be placed in specific locations with the [H] (e.g. \begin{table}[H])
\usepackage{hyperref} % For hyperlinks in the PDF

\usepackage{abstract} % Allows abstract customization
\renewcommand{\abstractnamefont}{\normalfont\bfseries} % Set the "Abstract" text to bold
\renewcommand{\abstracttextfont}{\normalfont\small\itshape} % Set the abstract itself to small italic text

\usepackage{titlesec} % Allows customization of titles
\renewcommand\thesection{\Roman{section}} % Roman numerals for the sections
\renewcommand\thesubsection{\Roman{subsection}} % Roman numerals for subsections
\titleformat{\section}[block]{\large\scshape\centering}{\thesection.}{1em}{} % Change the look of the section titles
\titleformat{\subsection}[block]{\large}{\thesubsection.}{1em}{} % Change the look of the section titles

\usepackage{fancyhdr} % Headers and footers
\pagestyle{fancy} % All pages have headers and footers
\fancyhead{} % Blank out the default header
\fancyfoot{} % Blank out the default footer
\fancyhead[C]{ Marzo 2016 $\bullet$ JumpVa $\bullet$ Análisis de la Aplicación} % Custom header text
\fancyfoot[RO,LE]{\thepage} % Custom footer text

%----------------------------------------------------------------------------------------
%	TITLE SECTION
%----------------------------------------------------------------------------------------

\title{\vspace{-15mm}\fontsize{24pt}{10pt}\selectfont\textbf{Mapa del Sitio}} % Article title

\author{
\large
\textsc{Alberto Amigo Alonso\textsubscript{20\%}}\\[2mm] % Your name
\textsc{Sergio Delgado Álvarez\textsubscript{20\%}}\\[2mm] % Your name
\textsc{Sergio García Prado\textsubscript{20\%}}\\[2mm] % Your name
\textsc{Oscar Fernández Angulo\textsubscript{20\%}}\\[2mm] % Your name
\textsc{Silvia Rodriguez Ares\textsubscript{20\%}}\\[2mm] % Your name
\normalsize Universidad de Valladolid \\ % Your institution
\vspace{-5mm}
}
\date{}

%----------------------------------------------------------------------------------------

\begin{document}

	\maketitle % Insert title

	\thispagestyle{fancy} % All pages have headers and footers

%----------------------------------------------------------------------------------------
%	ABSTRACT
%----------------------------------------------------------------------------------------

	\begin{abstract}
		\noindent Servicio web destinado a permitir a remitentes y destinatarios ofertar envíos para que los transportistas sean capaces de encontrarlos permitiendo a todos los usuarios monitorizarlos.
	\end{abstract}

%----------------------------------------------------------------------------------------
%	TEXT
%----------------------------------------------------------------------------------------
	\section{Mapa del sitio:}


		\begin{figure}[H]
			\centering
				\includegraphics[width=0.75\textwidth]{sitemap.png}
		\end{figure}

	\section{Guía de Uso}

		\paragraph{}
		Dado que la página web está hecha con Angular.js y se ha utilizado una única página raiz en la cual se van incluyendo las distintas partes según convenga, para probar la página hay que ejecutar \textbf(web/index.html).

		En esta vista hay dos botones verdes los cuales simulan lo que verá un transportistas o un cliente.

\end{document}
