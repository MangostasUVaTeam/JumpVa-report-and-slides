\documentclass[10pt, a4paper,spanish]{article}
\usepackage[utf8]{inputenc}

\usepackage{lipsum} % Package to generate dummy text throughout this template
\usepackage{varwidth}
\usepackage{hyperref}

\usepackage[T1]{fontenc} % Use 8-bit encoding that has 256 glyphs
\usepackage{microtype} % Slightly tweak font spacing for aesthetics

\usepackage[hmarginratio=1:1,top=32mm,columnsep=20pt]{geometry} % Document margins
\usepackage{multicol} % Used for the two-column layout of the document
\usepackage[hang, small,labelfont=bf,up,textfont=it,up]{caption} % Custom captions under/above floats in tables or figures
\usepackage{booktabs} % Horizontal rules in tables
\usepackage{float} % Required for tables and figures in the multi-column environment - they need to be placed in specific locations with the [H] (e.g. \begin{table}[H])
\usepackage{hyperref} % For hyperlinks in the PDF

\usepackage{lettrine} % The lettrine is the first enlarged letter at the beginning of the text
\usepackage{paralist} % Used for the compactitem environment which makes bullet points with less space between them

\usepackage{abstract} % Allows abstract customization
\renewcommand{\abstractnamefont}{\normalfont\bfseries} % Set the "Abstract" text to bold
\renewcommand{\abstracttextfont}{\normalfont\small\itshape} % Set the abstract itself to small italic text

\usepackage{titlesec} % Allows customization of titles
\renewcommand\thesection{\Roman{section}} % Roman numerals for the sections
\renewcommand\thesubsection{\Roman{subsection}} % Roman numerals for subsections
\titleformat{\section}[block]{\large\scshape\centering}{\thesection.}{1em}{} % Change the look of the section titles
\titleformat{\subsection}[block]{\large}{\thesubsection.}{1em}{} % Change the look of the section titles

\usepackage{fancyhdr} % Headers and footers
\pagestyle{fancy} % All pages have headers and footers
\fancyhead{} % Blank out the default header
\fancyfoot{} % Blank out the default footer
\fancyhead[C]{Sergio García Prado $\bullet$ Febrero 2016 $\bullet$ Propuesta Individual} % Custom header text
\fancyfoot[RO,LE]{\thepage} % Custom footer text

%----------------------------------------------------------------------------------------
%	TITLE SECTION
%----------------------------------------------------------------------------------------

\title{\vspace{-15mm}\fontsize{24pt}{10pt}\selectfont\textbf{Propuesta Individual}} % Article title

\author{
\large
\textsc{Sergio García Prado}\\[2mm] % Your name
\normalsize Universidad de Valladolid \\ % Your institution
\vspace{-5mm}
}
\date{}

%----------------------------------------------------------------------------------------

\begin{document}

	\maketitle % Insert title

	\thispagestyle{fancy} % All pages have headers and footers

%----------------------------------------------------------------------------------------
%	ABSTRACT
%----------------------------------------------------------------------------------------

	\begin{abstract}
		\noindent Página Web 
	\end{abstract}

		\section{Proposición de Compañeros del grupo de trabajo}
			
			\begin{center}
				\begin{varwidth}{\textwidth}
					\begin{itemize}
						\item Amigo Alonso, Alberto
						\item Delgado Álvarez, Sergio
						\item Fernández Angulo, Oscar
					\end{itemize}
				\end{varwidth}
			\end{center}
			
		\section{Descripción del Problema}
		
			\paragraph{}
			El transporte es una actividad del sector terciario, entendida como el desplazamiento de objetos o personas de un lugar (punto de origen) a otro (punto de destino) en un vehículo (medio o sistema de transporte) que utiliza una determinada infraestructura (red de transporte). Esta ha sido una de las actividades terciarias que mayor expansión ha experimentado a lo largo de los últimos dos siglos, debido a la industrialización; al aumento del comercio y de los desplazamientos humanos tanto a escala nacional como internacional; y los avances técnicos que se han producido y que han repercutido en una mayor rapidez, capacidad, seguridad y menor coste de los transportes. \cite{wikipedia_transporte}
		\section{Usuarios Objetivo}
		
			\paragraph{}
			Página Web

		\section{Borrador de la Solución}
		
			\paragraph{}
			Página Web


%Bibliographic references
\begin{thebibliography}{9}
\bibitem{wikipedia_transporte} 
Wikipedia. Transporte. \url{https://es.wikipedia.org/wiki/Transporte}
\end{thebibliography}
 

\end{document}
