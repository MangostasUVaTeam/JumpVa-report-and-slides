\documentclass[10pt, a4paper,spanish]{article}
\usepackage[utf8]{inputenc}

\usepackage{lipsum} % Package to generate dummy text throughout this template
\usepackage{varwidth}
\usepackage{hyperref}

\usepackage[T1]{fontenc} % Use 8-bit encoding that has 256 glyphs
\usepackage{microtype} % Slightly tweak font spacing for aesthetics

\usepackage[hmarginratio=1:1,top=32mm,columnsep=20pt]{geometry} % Document margins
\usepackage{multicol} % Used for the two-column layout of the document
\usepackage[hang, small,labelfont=bf,up,textfont=it,up]{caption} % Custom captions under/above floats in tables or figures
\usepackage{booktabs} % Horizontal rules in tables
\usepackage{float} % Required for tables and figures in the multi-column environment - they need to be placed in specific locations with the [H] (e.g. \begin{table}[H])
\usepackage{hyperref} % For hyperlinks in the PDF

\usepackage{lettrine} % The lettrine is the first enlarged letter at the beginning of the text
\usepackage{paralist} % Used for the compactitem environment which makes bullet points with less space between them

\usepackage{abstract} % Allows abstract customization
\renewcommand{\abstractnamefont}{\normalfont\bfseries} % Set the "Abstract" text to bold
\renewcommand{\abstracttextfont}{\normalfont\small\itshape} % Set the abstract itself to small italic text

\usepackage{titlesec} % Allows customization of titles
\renewcommand\thesection{\Roman{section}} % Roman numerals for the sections
\renewcommand\thesubsection{\Roman{subsection}} % Roman numerals for subsections
\titleformat{\section}[block]{\large\scshape\centering}{\thesection.}{1em}{} % Change the look of the section titles
\titleformat{\subsection}[block]{\large}{\thesubsection.}{1em}{} % Change the look of the section titles

\usepackage{fancyhdr} % Headers and footers
\pagestyle{fancy} % All pages have headers and footers
\fancyhead{} % Blank out the default header
\fancyfoot{} % Blank out the default footer
\fancyhead[C]{Sergio García Prado $\bullet$ Febrero 2016 $\bullet$ Propuesta Individual} % Custom header text
\fancyfoot[RO,LE]{\thepage} % Custom footer text

%----------------------------------------------------------------------------------------
%	TITLE SECTION
%----------------------------------------------------------------------------------------

\title{\vspace{-15mm}\fontsize{24pt}{10pt}\selectfont\textbf{Propuesta Individual}} % Article title

\author{
\large
\textsc{Sergio García Prado}\\[2mm] % Your name
\normalsize Universidad de Valladolid \\ % Your institution
\vspace{-5mm}
}
\date{}

%----------------------------------------------------------------------------------------

\begin{document}

	\maketitle % Insert title

	\thispagestyle{fancy} % All pages have headers and footers

%----------------------------------------------------------------------------------------
%	ABSTRACT
%----------------------------------------------------------------------------------------

	\begin{abstract}
		\noindent Página Web 
	\end{abstract}
	
%----------------------------------------------------------------------------------------
%	TEXT
%----------------------------------------------------------------------------------------

		\section{Proposición de Compañeros del grupo de trabajo}
			
			\begin{center}
				\begin{varwidth}{\textwidth}
					\begin{itemize}
						\item Amigo Alonso, Alberto
						\item Delgado Álvarez, Sergio
						\item Fernández Angulo, Oscar
					\end{itemize}
				\end{varwidth}
			\end{center}
			
		\section{Descripción del Problema}
		
			\paragraph{}
			El transporte es una actividad del sector terciario, entendida como el desplazamiento de objetos o personas de un lugar (punto de origen) a otro (punto de destino) en un vehículo (medio o sistema de transporte) que utiliza una determinada infraestructura (red de transporte). Esta ha sido una de las actividades terciarias que mayor expansión ha experimentado a lo largo de los últimos dos siglos, debido a la industrialización; al aumento del comercio y de los desplazamientos humanos tanto a escala nacional como internacional; y los avances técnicos que se han producido y que han repercutido en una mayor rapidez, capacidad, seguridad y menor coste de los transportes. \cite{wikipedia_transporte}
			
			\paragraph{}
			Nosotros nos centraremos en el transporte de mercancías para la realización de esta propuesta. En este sector existen tres roles bien diferenciados: el \textbf{remitente}, el \textbf{transportista} y el \textbf{destinatario}. El modelo de negocio actual se basa en empresas intermediarias denominadas \textit{agencias de transporte} cuya labor es poner en contacto remitentes con transportistas que lleven la mercancía hasta su destino. 
			
			\paragraph{}
			Dependiendo de cómo se contrate el transporte existen dos formas de pago: por parte del remitente, del destinatario o respecto de una determinada proporción. Actualmente el método más utilizado es el de pago por parte del destinatario pero aún así existen casos especiales en los que no. También existen dos métodos de calcular el precio del transporte: aplicando una tarifa fija (lo cual muchas veces infla los costes de manera innecesaria) o teniendo en cuenta la distancia recorrida desde el origen hasta el destino. Uno de los problemas actuales es el de calcular penalizaciones y recompensas por el tiempo de transporte así como el periodo de tiempo que transcurre durante la carga y descarga de la mercancía.
			
			\paragraph{}
			La solución que se propone es crear un servicio encargado de conectar a transportistas con remitentes y destinatarios de manera eficiente y automática tratando de reducir los costes entre intermediarios, dando lugar a un mayor beneficio para el transportista y un menor coste para el remitente/destinatario. En la siguiente sección se describirá más detalladamente tanto las acciones como la descripción de cada uno de los usuarios objetivo.
			
			\paragraph{}
			El servicio seguirá una idea similar a la de Uber \copyright\  con el transporte de pasajeros pero aplicado al campo de las mercancías (siempre siguiendo la vigente normativa). Ya existen algunas start-ups que pretenden hacerse un hueco en este nicho de mercado entre las que destacan Convoy \copyright\ y Trucker Path \copyright\ las cuales se basan en una aplicación para sistemas móviles. \cite{expansion_uber_transporte}
			
			
			
		\section{Usuarios Objetivo}
		
			\paragraph{}
			Los usuarios objetivo del servicio serán los siguientes:
			
			\begin{itemize}
				\item{\textbf{Remitente}}
				\newline
				Página Web
				
				\item{\textbf{Transportistas}}
				\newline
				Página Web

				
				\item{\textbf{Destinatario}}
				\newline
				Página Web

			\end{itemize}
		\section{Borrador de la Solución}
		
			\paragraph{}
			Página Web

%----------------------------------------------------------------------------------------
%	Bibliographic references
%----------------------------------------------------------------------------------------
	\begin{thebibliography}{9}
	
		\bibitem{wikipedia_transporte} 
		Wikipedia. Transporte. \url{https://es.wikipedia.org/wiki/Transporte}

		\bibitem{expansion_uber_transporte} 
		Expansión. Se busca el Uber del transporte de mercancías. \url{http://www.expansion.com/economia-digital/2015/10/29/5630ef8e46163f932a8b45d9.html}
	
	\end{thebibliography}
 

\end{document}
